\documentclass[a4paper]{article}
\usepackage{graphicx} % Required for inserting images
\usepackage{todonotes}
\usepackage{geometry}
\geometry{a4paper, top=3cm, bottom=3cm, left=3.5cm, right=3.5cm} % heightrounded, bindingoffset=5mm}

\title{Progetto Basi di Dati 2023-24 \\
“UNIGE SOCIAL SPORT” \\
Parte I}
\author{[Numero Gruppo]
\\
\and [Nome, Cognome e Matricola dei componenti]}
\date{}

\begin{document}

\maketitle

\todo[inline]{Per i gruppi che hanno effettuato la peer review le sezioni 1. e 2. devono essere conse-gnate in versione finale  e consistenti con la progettazione relativa ai passi successivi.}


\section{Requisiti Ristrutturati}
\todo[inline]{Riportare in questa sezione i requisiti ristrutturati in modo da eliminare ambiguità, evidenziando le modifiche effettuate}

\section{Progetto Concettuale}
\subsection{SCHEMA ENTITY RELATIONSHIP}
\todo[inline]{Riportare in questa sezione il diagramma ER, indicando anche il tipo delle gerarchie di gneralizzazione}

\subsection{DIZIONARIO DATI - DOMINI DEGLI ATTRIBUTI}
\subsection{VINCOLI NON ESPRIMIBILI NEL DIAGRAMMA}

\section{Progetto Logico}
\subsection{SCHEMA ER RISTRUTTURATO}
\subsection{DOMINI DEGLI ATTRIBUTI}
\todo[inline]{eventuali modifiche dei domini degli attributi e informazioni sui domini di eventuali attributi introdotti}
\subsection{VINCOLI}
\todo[inline]{modifiche all'elenco di vincoli del modello concettuale (nuovi vincoli, eventuali vincoli eliminati o modificati)}
\subsection{RISTRUTTURAZIONE GERARCHIE}
\todo[inline]{discussione delle scelte fatte per eliminare le gerarchie di generalizzazione}

\subsection{SCHEMA LOGICO}
\todo[inline]{schema logico relazionale nella notazione vista a lezione}

\subsection{VERIFICA DI QUALITÀ DELLO SCHEMA}
\todo[inline]{verifica delle forme normali e eventuali ottimizzazioni applicate tenendo in considerazione il carico di lavoro}


\subsection{SCHEMA SQL IN FORMA GRAFICA}
\todo[inline]{diagramma che visualizza lo script SQL di creazione dello schema in forma grafica ottenuto con DataGrip (vedi Aulabweb per come crearla dallo script)}

\end{document}
